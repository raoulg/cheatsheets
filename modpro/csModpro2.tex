%%%%%%%%%%%%%%%%%%%%%%%%%%%%%%%%%%%%%%%%%
% Cheatsheet
% LaTeX Template
% Version 1.0 (12/12/15)
% This template has been downloaded from:
% http://www.LaTeXTemplates.com
% Original author:
% Michael Müller (https://github.com/cmichi/latex-template-collection) with
% extensive modifications by Vel (vel@LaTeXTemplates.com)
% License:
% The MIT License (see included LICENSE file)
%%%%%%%%%%%%%%%%%%%%%%%%%%%%%%%%%%%%%%%%%
% cheatsheet voor Modelleren en Programmeren
% gemaakt door R.Grouls

\documentclass[10pt]{scrartcl} % 10pt font size
\usepackage[utf8]{inputenc} % Required for inputting international characters
\usepackage[T1]{fontenc} % Output font encoding for international characters
\usepackage[margin=0pt, landscape]{geometry} % Page margins and orientation
\usepackage{graphicx} % Required for including images
\usepackage{color} % Required for color customization
\usepackage{url} % Required for the \url command to easily display URLs
\usepackage{textcomp}
\usepackage[ % This block contains information used to annotate the PDF
colorlinks=false, 
pdftitle={Cheatsheet Wiskunde I}, 
pdfauthor={R.Grouls}, 
]{hyperref}

\setlength{\unitlength}{1mm} % Set the length that numerical units are measured in
\setlength{\parindent}{0pt} % Stop paragraph indentation

\renewcommand{\dots}{\ \dotfill{}\ } % Fills in the right amount of dots
\newcommand{\command}[2]{#1~\dotfill{}~#2\\} % Custom command for adding a shorcut  
\newcommand{\raw}{$\rightarrow$} % shortcut voor arrows
\usepackage[scaled=.8]{beramono}
\newcommand{\sectiontitle}[1]{\vfill \textbf{#1}\\}

%----------------------------------------------------------------------------------------
\begin{document}
\begin{picture}(297,210) % Create a container for the page content
%----------------------------------------------------------------------------------------
%	TITLE SECTION 
%----------------------------------------------------------------------------------------
\put(10,200){% Position on the page to put the title
\begin{minipage}[t]{210mm} % The size and alignment of the title
\section*{ModProg -- II} % Title
\end{minipage}
}

%----------------------------------------------------------------------------------------
%	FIRST COLUMN SPECIFICATION
%----------------------------------------------------------------------------------------
\put(10,190){% Divide the page
    \begin{minipage}[t]{85mm} % Create a box to house text
        \sectiontitle{H8.1-2, 10.1 (objecten en klassen)}
        \command{object}{groepje variabelen}
        \command{klasse-definitie}{member-var, methoden, properties}
        \command{subklasse}{erft vars uit superklasse}
        \command{syntax subklasse}{\texttt{class Subklasse : Superklasse}}
        \command{meerdere klassen}{voeg "zelfgemaakte" objecten toe}
        \command{\texttt{new}}{reserveert geheugen, roept constr aan}
        \command{constructormethode}{ideaal voor beginwaarden}
        \command{constructormet.}{zelfde naam als klasse}
        \command{constructormet.}{roept constr superklasse aan}
        \command{constructormet.}{geen resultaattype, zelfs niet \texttt{void}}
        \command{declaratie array}{\texttt{int[] list;}}
        \command{constructie array}{\texttt{list=new int [5];}}
        \command{\texttt{x|=y}}{\texttt{x = x | y}}
%----------------------------------------------------------------------------------------
        \sectiontitle{H8.3, 9.3 (animatie en chars)}
        \command{\texttt{System.Threading}}{elke \texttt{Thread} runt simultaan}
        \command{\texttt{Thread}}{ideaal voor animaties}
        \command{\texttt{Thread}}{wacht niet tot methode is afgelopen}
        \command{\texttt{new Thread(methode)}}{maakt \texttt{Thread} voor \texttt{methode}}
        \command{\texttt{Thread}}{kent methode \texttt{Start()}}
        \command{\texttt{Start()}}{roept meth aan en keert direct terug}
        \command{\texttt{Thread}}{kent methode \texttt{Stop()}}
        \command{\texttt{Thread}}{methode \texttt{Sleep(n)} wacht \texttt{n} ms}
        \command{\texttt{while(true)}}{loopt constant door}
        \command{\texttt{while(var)}}{\texttt{var} aanpasbaar in andere \texttt{Thread}}
        \command{\texttt{null}}{verwijst naar niks}
        \command{primitieve types}{\texttt{int,double,bool,char}}
        \command{\texttt{string}}{klasse met methods en props}
        \command{string-methodes}{\texttt{Length,Equals,Concat,ToUpper}}
        \command{string-methodes}{\texttt{Substring(int start)}}
        \command{string-methodes}{\texttt{Substring(int start, int aantal)}}
        \command{string-methodes}{\texttt{IndexOf(string s)}}
        \texttt{string x = "hamburger"}\\
        \command{\texttt{x.Substring(3)}}{\texttt{burger}}
        \command{\texttt{x.Substring(3,4)}}{\texttt{burg}}
        \command{\texttt{x.IndexOf("burg")}}{\texttt{3}}
        \command{\texttt{x.IndexOf("dfsa")}}{\texttt{-1}}
        \command{\texttt{char} is geordend}{\texttt{'A'<'Z'}}
        \command{\texttt{char} converteert naar \texttt{int}}{\texttt{int n='A'}}
        \command{\texttt{int} converteert naar \texttt{char}}{\texttt{char c=65}}
        \command{\texttt{'\textbackslash n', '\textbackslash t'}}{twee tekens, toch \'e\'en character}

%----------------------------------------------------------------------------------------
    \end{minipage} % End the first column of text
} % End the first division of the page
%----------------------------------------------------------------------------------------
%	SECOND COLUMN SPECIFICATION 
%----------------------------------------------------------------------------------------
\put(105,200){% Divide the page
\begin{minipage}[t]{85mm} % Create a box to house text
        \sectiontitle{H9.4}
        \command{\texttt{get}}{}
        \sectiontitle{H9.4}
        \sectiontitle{H10.4}
        \sectiontitle{Satsolver opdracht (HC12 en 14)}
        \sectiontitle{H12.1}
        \sectiontitle{H11.1}
%----------------------------------------------------------------------------------------
\end{minipage} % End the second column of text
} % End the second division of the page
%----------------------------------------------------------------------------------------
%	THIRD COLUMN SPECIFICATION 
%----------------------------------------------------------------------------------------
\put(200,200){% Divide the page
\begin{minipage}[t]{85mm} % Create a box to house tex
%	FOOTNOTE
%----------------------------------------------------------------------------------------
\linethickness{0.2mm} % Thickness of the footer line
{\color{black}\line(1,0){30}} % Print the line with a custom color
\footnotesize{\\
Created by: \\
Raoul Grouls\\
Notatie: alle C\#-code is herkenbaar aan het \texttt{typewriter-font}
2017
}
%----------------------------------------------------------------------------------------
\end{minipage} % End the third column of text
} % End the third division of the page
\end{picture} % End the container for the entire page
\end{document}
