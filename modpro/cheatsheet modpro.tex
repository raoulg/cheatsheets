%%%%%%%%%%%%%%%%%%%%%%%%%%%%%%%%%%%%%%%%%
% Cheatsheet
% LaTeX Template
% Version 1.0 (12/12/15)
%
% This template has been downloaded from:
% http://www.LaTeXTemplates.com
%
% Original author:
% Michael Müller (https://github.com/cmichi/latex-template-collection) with
% extensive modifications by Vel (vel@LaTeXTemplates.com)
%
% License:
% The MIT License (see included LICENSE file)
%
%%%%%%%%%%%%%%%%%%%%%%%%%%%%%%%%%%%%%%%%%

%----------------------------------------------------------------------------------------
%	PACKAGES AND OTHER DOCUMENT CONFIGURATIONS
%----------------------------------------------------------------------------------------

\documentclass[10pt]{scrartcl} % 11pt font size

\usepackage[utf8]{inputenc} % Required for inputting international characters
\usepackage[T1]{fontenc} % Output font encoding for international characters

\usepackage[margin=0pt, landscape]{geometry} % Page margins and orientation

\usepackage{graphicx} % Required for including images
\usepackage{color} % Required for color customization
\usepackage{url} % Required for the \url command to easily display URLs
\usepackage{textcomp}

\usepackage[ % This block contains information used to annotate the PDF
colorlinks=false, 
pdftitle={Cheatsheet Wiskunde I}, 
pdfauthor={R.Grouls}, 
]{hyperref}

\setlength{\unitlength}{1mm} % Set the length that numerical units are measured in
\setlength{\parindent}{0pt} % Stop paragraph indentation

\renewcommand{\dots}{\ \dotfill{}\ } % Fills in the right amount of dots

\newcommand{\command}[2]{#1~\dotfill{}~#2\\} % Custom command for adding a shorcut
\newcommand{\raw}{$\rightarrow$}
%\newcommand{\sectiontitle}[1]{\paragraph{#1} \ \\} % Custom command for subsection titles
\newcommand{\sectiontitle}[1]{\vfill \textbf{#1}\\}

%----------------------------------------------------------------------------------------

\begin{document}

\begin{picture}(297,210) % Create a container for the page content

%----------------------------------------------------------------------------------------
%	TITLE SECTION 
%----------------------------------------------------------------------------------------

\put(10,200){ % Position on the page to put the title
\begin{minipage}[t]{210mm} % The size and alignment of the title
\section*{ModProg -- I} % Title
\end{minipage}
}

%----------------------------------------------------------------------------------------
%	FIRST COLUMN SPECIFICATION
%----------------------------------------------------------------------------------------

\put(10,190){ % Divide the page
\begin{minipage}[t]{85mm} % Create a box to house text
\sectiontitle{1.1-2: Programmeren}
\command{computer}{processor en geheugen}
\command{opdracht}{processor verandert geheugen}
\command{methode}{opdr.\raw methode}
\command{klasse}{opdr.\raw meth.\raw klasse}
\command{namespace}{opdr.\raw meth.\raw klas.\raw namespace}
\command{variabele}{geheugenplaats met een naam}
\command{object}{variabelen\raw object}
\sectiontitle{1.3 Programmeerparadigmas}
\command{declaratief}{functie-based}
\command{declaratief-functioneel}{legt funct. verbanden}
\command{declaratief-logisch}{propositielogica}
\command{imperatief}{opdracht-based (gebiedend)}
\command{enkel imperatief}{bv: Fortran, Basic}
\command{procedureel}{imperatief + methoden}
\command{procedureel}{bv: Python, PHP}
\command{object-geori\"enteerd}{procedureel + objecten}
\command{object-geori\"enteerd}{bv: C\#, Java, C++}
\sectiontitle{1.5 Vertalen}
\command{assembler}{vertaalt 1 sourcecode voor 1 processor}
\command{compiler}{vertaalt 1 sourcecode voor alle processors}
\command{interpreter}{voert sourcecode direct uit}
\command{Java}{compiler+interpreter \raw bytecode}
\command{C\#}{compiler+compiler \raw intermediate language}
\sectiontitle{2. C\# programma's}
\command{opbouw broncode}{class\{methode\{opdracht\}\}\}}
\command{class}{public/private}
\command{class-members}{(public/private)(static)(type/void)}
\command{method body}{declaratie/opdracht}
\command{declaratie}{reserveert geheugen}
\command{declaratie}{\texttt{int x; string naam;}}
\command{opdracht}{aanroep/toekenning}
\command{opdracht-aanroep}{roept andere methode aan}
\command{opdracht-aanroep}{\texttt{Console.Writeline(x);}}
\command{opdracht-toekenning}{verandert het geheugen}
\command{opdracht-toekenning}{\texttt{x = 1; naam.Text = "a";}}
\command{property}{eigenschap van een object}
\command{subklasse}{subversie van bestaande klasse}
\command{subklasse}{\texttt{class scherm : Form}}
\command{constructormethode}{maakt \texttt{new} object}
\command{constructormethode}{methode met naam van class}
%----------------------------------------------------------------------------------------
\end{minipage} % End the first column of text
} % End the first division of the page
%----------------------------------------------------------------------------------------
%	SECOND COLUMN SPECIFICATION 
%----------------------------------------------------------------------------------------
\put(105,200){ % Divide the page
\begin{minipage}[t]{85mm} % Create a box to house text
\command{\texttt{this}}{object dat door methode bewerkt wordt}
\command{\texttt{public}}{bruikbaar in andere klasses}
\command{property van object}{\texttt{naam.Length}}
\command{niet-static methoden}{werkt op object}
\command{niet-static methoden}{\texttt{naam.ToUpper();}}
\command{static}{class ipv object}
\command{static properties}{\texttt{Color.Yellow}}
\command{static methoden}{\texttt{Console.Writeline(x)}}
\command{\texttt{Main}}{altijd \texttt{static}}
\command{constructor}{nooit \texttt{static}}
\command{eventhandler}{\texttt{this.Paint+=teken;}}
\command{eventhandler}{toekenning (geen aanroep)}
\sectiontitle{3. Tekenen}
\command{library}{\texttt{using System.Drawing;}}
\command{\texttt{Paint}-event}{\texttt{(object o, PaintEventArgs pea)}}
\command{\texttt{pea.Graphics}}{property van \texttt{pea}}
\command{\texttt{pea.Graphics.DrawLine()}}{tekentlijnen}
\command{\texttt{DrawLine,Rectangle,Ellipse}}{gebruiken \texttt{Pen}}
\command{\texttt{FillRectangle,Ellipse}}{gebruiken \texttt{Brush}}
\command{\texttt{DrawString}}{gebruikt \texttt{Brush}}
\sectiontitle{3.3-5 Berekeningen}
\command{expressie}{alle code met een waarde}
\command{expressie}{int, string, object-waarde,constructor \texttt{new}}
\command{\texttt{new Form()}}{heeft \texttt{Form}-object als waarde}
\command{expressie}{kun je \textit{uitrekenen} en heeft \textit{waarde}}
\command{opdracht}{kun je \textit{uitvoeren} en heeft \textit{effect}}
\command{expressies}{kunnen deel uitmaken van opdracht}
\command{modulo \%}{8\%3=2}
\command{\texttt{const}}{onaanpasbare declaratie}
\command{\texttt{var}}{declaratie met automatische typebepaling}



%----------------------------------------------------------------------------------------
\end{minipage} % End the second column of text
} % End the second division of the page
%----------------------------------------------------------------------------------------
%	THIRD COLUMN SPECIFICATION 
%----------------------------------------------------------------------------------------
\put(200,200){ % Divide the page
\begin{minipage}[t]{85mm} % Create a box to house tex

\sectiontitle{}


\sectiontitle{}

%	FOOTNOTE
%----------------------------------------------------------------------------------------


\linethickness{0.2mm} % Thickness of the footer line
{\color{black}\line(1,0){30}} % Print the line with a custom color

\footnotesize{
Created by Raoul Grouls, 2017\\

				
Released under the MIT license.}

%----------------------------------------------------------------------------------------

\end{minipage} % End the third column of text
} % End the third division of the page
\end{picture} % End the container for the entire page

%----------------------------------------------------------------------------------------

\end{document}
