%%%%%%%%%%%%%%%%%%%%%%%%%%%%%%%%%%%%%%%%%
% Cheatsheet
% LaTeX Template
% Version 1.0 (12/12/15)
% This template has been downloaded from:
% http://www.LaTeXTemplates.com
% Original author:
% Michael Müller (https://github.com/cmichi/latex-template-collection) with
% extensive modifications by Vel (vel@LaTeXTemplates.com)
% License:
% The MIT License (see included LICENSE file)
%%%%%%%%%%%%%%%%%%%%%%%%%%%%%%%%%%%%%%%%%
% cheatsheet voor Modelleren en Programmeren
% gemaakt door R.Grouls

\documentclass[10pt]{scrartcl} % 11pt font size
\usepackage[utf8]{inputenc} % Required for inputting international characters
\usepackage[T1]{fontenc} % Output font encoding for international characters
\usepackage[margin=0pt, landscape]{geometry} % Page margins and orientation
\usepackage{graphicx} % Required for including images
\usepackage{color} % Required for color customization
\usepackage{url} % Required for the \url command to easily display URLs
\usepackage{textcomp}
\usepackage[ % This block contains information used to annotate the PDF
colorlinks=false, 
pdftitle={Cheatsheet Wiskunde I}, 
pdfauthor={R.Grouls}, 
]{hyperref}

\setlength{\unitlength}{1mm} % Set the length that numerical units are measured in
\setlength{\parindent}{0pt} % Stop paragraph indentation

\renewcommand{\dots}{\ \dotfill{}\ } % Fills in the right amount of dots
\newcommand{\command}[2]{#1~\dotfill{}~#2\\} % Custom command for adding a shorcut

\newcommand{\raw}{$\rightarrow$} % shortcut voor arrows
\newcommand{\sectiontitle}[1]{\vfill \textbf{#1}\\}

%----------------------------------------------------------------------------------------
\begin{document}
\begin{picture}(297,210) % Create a container for the page content
%----------------------------------------------------------------------------------------
%	TITLE SECTION 
%----------------------------------------------------------------------------------------
\put(10,200){ % Position on the page to put the title
\begin{minipage}[t]{210mm} % The size and alignment of the title
\section*{ModProg -- I} % Title
\end{minipage}
}

%----------------------------------------------------------------------------------------
%	FIRST COLUMN SPECIFICATION
%----------------------------------------------------------------------------------------
\put(10,190){ % Divide the page
\begin{minipage}[t]{85mm} % Create a box to house text
\sectiontitle{1.1-2: Programmeren}
\command{computer}{processor en geheugen}
\command{opdracht}{processor verandert geheugen}
\command{methode}{opdr.\raw methode}
\command{klasse}{opdr.\raw meth.\raw klasse}
\command{namespace}{opdr.\raw meth.\raw klas.\raw namespace}
\command{variabele}{geheugenplaats met een naam}
\command{object}{variabelen\raw object}
\command{objecten}{hebben als type een \texttt{class}}
%\sectiontitle{1.3 Programmeerparadigmas}
%\command{declaratief}{functie-based}
%\command{declaratief-functioneel}{legt funct. verbanden}
%\command{declaratief-logisch}{propositielogica}
%\command{imperatief}{opdracht-based (gebiedend)}
%\command{enkel imperatief}{bv: Fortran, Basic}
%\command{procedureel}{imperatief + methoden}
%\command{procedureel}{bv: Python, PHP}
%\command{object-geori\"enteerd}{procedureel + objecten}
%\command{object-geori\"enteerd}{bv: C\#, Java, C++}
%\sectiontitle{1.5 Vertalen}
%\command{assembler}{vertaalt 1 sourcecode voor 1 processor}
%\command{compiler}{vertaalt 1 sourcecode voor alle processors}
%\command{interpreter}{voert sourcecode direct uit}
%\command{Java}{compiler+interpreter \raw bytecode}
%\command{C\#}{compiler+compiler \raw intermediate language}
\sectiontitle{2. C\# programma's}
\command{opbouw broncode}{class\{methode\{opdracht\}\}\}}
\command{class}{public/private}
\command{class-members}{(public/private)(static)(type/void)}
\command{method body}{declaratie/opdracht}
\command{declaratie}{reserveert geheugen}
\command{declaratie}{\texttt{int x; string naam;}}
\command{opdracht}{aanroep/toekenning}
\command{opdracht-aanroep}{roept andere methode aan}
\command{opdracht-aanroep}{\texttt{Console.Writeline(x);}}
\command{opdracht-toekenning}{verandert het geheugen}
\command{opdracht-toekenning}{\texttt{x = 1; naam.Text = "a";}}
\command{property}{eigenschap van een object}
\command{subklasse}{subversie van bestaande klasse}
\command{subklasse}{\texttt{class scherm : Form}}
\command{constructormethode}{maakt \texttt{new} object}
\command{constructormethode}{methode met naam van class}
\command{\texttt{this}}{object dat door methode bewerkt wordt}
\command{\texttt{public}}{bruikbaar in andere klasses}
\command{property van object}{\texttt{naam.Length}}
\command{niet-static methoden}{werkt op object}
\command{niet-static methoden}{\texttt{naam.ToUpper();}}
\command{static}{class ipv object}
\command{static properties}{\texttt{Color.Yellow}}
\command{static methoden}{\texttt{Console.Writeline(x)}}
\command{\texttt{Main}}{altijd \texttt{static}}
\command{constructor}{nooit \texttt{static}}
\command{eventhandler}{\texttt{this.Paint+=teken;}}
\command{eventhandler}{toekenning (geen aanroep)}
\sectiontitle{3. Tekenen}
\command{library}{\texttt{using System.Drawing;}}
\command{\texttt{Paint}-event}{\texttt{(object o, PaintEventArgs pea)}}
\command{\texttt{pea.Graphics}}{property van \texttt{pea}}
\command{\texttt{pea.Graphics.DrawLine()}}{tekent lijnen}
%----------------------------------------------------------------------------------------
\end{minipage} % End the first column of text
} % End the first division of the page
%----------------------------------------------------------------------------------------
%	SECOND COLUMN SPECIFICATION 
%----------------------------------------------------------------------------------------
\put(105,200){ % Divide the page
\begin{minipage}[t]{85mm} % Create a box to house text
\command{\texttt{DrawLine,Rectangle,Ellipse}}{gebruiken \texttt{Pen}}
\command{\texttt{FillRectangle,Ellipse}}{gebruiken \texttt{Brush}}
\command{\texttt{DrawString}}{gebruikt \texttt{Brush}}
\sectiontitle{3.3-5 Berekeningen}
\command{expressie}{alle code met een waarde}
\command{expressie}{int, string, object-waarde,constructor \texttt{new}}
\command{\texttt{new Form()}}{heeft \texttt{Form}-object als waarde}
\command{expressie}{kun je \textit{uitrekenen} en heeft \textit{waarde}}
\command{opdracht}{kun je \textit{uitvoeren} en heeft \textit{effect}}
\command{expressies}{kunnen deel uitmaken van opdracht}
\command{modulo \%}{8\%3=2}
\command{\texttt{const}}{onaanpasbare declaratie}
\command{\texttt{var}}{declaratie met automatische typebepaling}
\command{\texttt{void}-methode}{geldt als opdracht}
\command{methode m\'et returntype}{geldt als expressie}
\sectiontitle{4 Variabelen}
\command{\texttt{sbyte}}{[-128,127] (1 byte, $2^8$)}
\command{\texttt{short}}{2 bytes ($2^{16}$, ook -)}
\command{\texttt{int}}{4 bytes ($2^{64}$, ook -)}
\command{\texttt{long}}{8 bytes ($2^{128}$, ook -)}
\command{\texttt{byte,ushort,uint,ulong}}{zonder -}
\command{\texttt{float}}{4 bytes met punt}
\command{\texttt{double}}{8 bytes met punt}
\command{\texttt{decimal}}{16 bytes met punt}
\command{opdrachten}{veranderen variabelen}
\command{methoden}{bewerken objecten}
\command{twee object-variabelen}{\texttt{struct} en \texttt{class}}
\command{bij \texttt{struct}}{toekenningen reserveren geheugenruimte}
\command{bij \texttt{class}}{toekenningen verwijzen naar objecten}
\command{\texttt{class} object verandert}{impact op alle verwijzingen}
\sectiontitle{4.4 typeringen}
\texttt{int i; double d;}\\
\command{\texttt{d = i;}}{coverteert \texttt{int} naar \texttt{double}}
\command{\texttt{i = d;}}{\texttt{ERROR}}
\command{\texttt{i = (int) d;}}{cast \texttt{double} naar \texttt{int}}
\sectiontitle{5 Interactie}
\command{geneste namespace}{\texttt{System.Windows.Forms} in \texttt{System}}
\command{\texttt{Forms}}{bevat o.a. klassen \texttt{TextBox} en \texttt{Button}}
\command{\texttt{Click}}{event-property van \texttt{Button}}
\command{string naar double}{\texttt{double.Parse(invoer.Text);}}
\command{double naar string}{\texttt{d.ToString();}}
\sectiontitle{6.2-5: Herhaling}
\command{\texttt{bool waarde;}}{waarde is \texttt{true} of \texttt{false}}
%----------------------------------------------------------------------------------------
\end{minipage} % End the second column of text
} % End the second division of the page
%----------------------------------------------------------------------------------------
%	THIRD COLUMN SPECIFICATION 
%----------------------------------------------------------------------------------------
\put(200,200){ % Divide the page
\begin{minipage}[t]{85mm} % Create a box to house tex
\command{\texttt{while(waarde)}}{herhaling bij \texttt{true}}
\command{\texttt{while(waarde)}}{stopt bij \texttt{false}}
\command{vergelijkings-operatoren}{ <, <=, >, >=, ==, !=}
\command{\texttt{<, ==, !=}}{kleiner dan, is gelijk aan, ongelijk aan}
\command{Logische operatoren}{\texttt{\&\&}(en), \texttt{||}(of), \texttt{!}(niet)}
\command{\texttt{i++}}{\texttt{i} wordt opgehoogd}
\command{\texttt{for(i=0;i<10;i++)}}{herhaalt 10x}
\command{niet uitgevoerde herhaling}{\texttt{while(1==0)}}
\command{oneindige herhaling}{\texttt{while (1==1)}}
\sectiontitle{9. Array's}
\command{array}{object met lijst variabelen van \'e\'en type}
\command{array declaratie}{\texttt{int [] k;}}
\command{array toekenningsopdracht}{\texttt{k = new int[5];}}
\command{array index}{array's start at \texttt{0}}
\command{array waarden}{\texttt{k[0] = 12;}}
\command{\texttt{l.Length}}{\texttt{5}}
\command{\texttt{for(i=0;i<l.Length;i++)\{lijst[i]=12\}}}{alles 12}
\command{array van objecten}{\texttt{Button [] knoppen;}}
\command{array 2-dimensionaal}{int [,] m;}
\command{array van arrays}{\texttt{int[][] n;}}
\command{\texttt{n[1]=new int[5];n[2]=new int[10]}}{lengte 5 en 10}
\command{overloaded methoden}{alleen andere input-parameters}
%	FOOTNOTE
%----------------------------------------------------------------------------------------
\linethickness{0.2mm} % Thickness of the footer line
{\color{black}\line(1,0){30}} % Print the line with a custom color
\footnotesize{\\
Created by: \\
Raoul Grouls, Casper van Laar\\
Notatie: alle C\#-code is herkenbaar aan het \texttt{typewriter-font}
2017
}

%----------------------------------------------------------------------------------------
\end{minipage} % End the third column of text
} % End the third division of the page
\end{picture} % End the container for the entire page
\end{document}
