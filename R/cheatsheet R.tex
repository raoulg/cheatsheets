%%%%%%%%%%%%%%%%%%%%%%%%%%%%%%%%%%%%%%%%%
% Cheatsheet
% LaTeX Template
% Version 1.0 (12/12/15)
%
% This template has been downloaded from:
% http://www.LaTeXTemplates.com
%
% Original author:
% Michael Müller (https://github.com/cmichi/latex-template-collection) with
% extensive modifications by Vel (vel@LaTeXTemplates.com)
%
% License:
% The MIT License (see included LICENSE file)
%
%%%%%%%%%%%%%%%%%%%%%%%%%%%%%%%%%%%%%%%%%

%----------------------------------------------------------------------------------------
%	PACKAGES AND OTHER DOCUMENT CONFIGURATIONS
%----------------------------------------------------------------------------------------

\documentclass[10pt]{scrartcl} % 11pt font size

\usepackage[utf8]{inputenc} % Required for inputting international characters
\usepackage[T1]{fontenc} % Output font encoding for international characters

\usepackage[margin=0pt, landscape]{geometry} % Page margins and orientation

\usepackage{graphicx} % Required for including images
\usepackage{color} % Required for color customization
\usepackage{url} % Required for the \url command to easily display URLs

\usepackage[ % This block contains information used to annotate the PDF
colorlinks=false, 
pdftitle={Cheatsheet}, 
pdfauthor={Raoul Grouls}, 
pdfsubject={shortcuts voor R}, 
pdfkeywords={R, Cheatsheet}
]{hyperref}

\setlength{\unitlength}{1mm} % Set the length that numerical units are measured in
\setlength{\parindent}{0pt} % Stop paragraph indentation

\renewcommand{\dots}{\ \dotfill{}\ } % Fills in the right amount of dots

\newcommand{\command}[2]{\texttt{#1}~\dotfill{}~#2\\} % Custom command for adding a shorcut

%\newcommand{\sectiontitle}[1]{\paragraph{#1} \ \\} % Custom command for subsection titles
\newcommand{\sectiontitle}[1]{\vfill \textbf{#1}\\}

%----------------------------------------------------------------------------------------

\begin{document}

\begin{picture}(297,210) % Create a container for the page content

%----------------------------------------------------------------------------------------
%	TITLE SECTION 
%----------------------------------------------------------------------------------------

\put(10,200){ % Position on the page to put the title
\begin{minipage}[t]{210mm} % The size and alignment of the title
\section*{Cheatsheet R -- I} % Title
\end{minipage}
}

%----------------------------------------------------------------------------------------
%	FIRST COLUMN SPECIFICATION
%----------------------------------------------------------------------------------------

\put(10,190){ % Divide the page
\begin{minipage}[t]{85mm} % Create a box to house text

%----------------------------------------------------------------------------------------
%	HEADING ONE
%----------------------------------------------------------------------------------------

\sectiontitle{Basics}
\command{x <- y}{Asign to variable}
\command{c(x,y,z)}{concatenate vector}
\command{x\textasciicircum 2,sqrt(x)}{x squared, square root x}
\command{abs(x)}{absolute value of x}
\command{?$f$}{call help for function $f$}
\command{args($f$)}{show arguments for function $f$}
%----------------------------------------------------------------------------------------
%	HEADING TWO
%----------------------------------------------------------------------------------------	
\sectiontitle{Workspaces and Files}			
\command{getwd()}{get current working directory}
\command{setwd()}{set current working directory}
\command{ls()}{list objects in workingspace}
\command{list.files(),dir()}{list all files}
\command{dir.create("name")}{create dir "name" in pwd}
\command{file.create("file.R")}{create file.R}
\command{file.exist("file.R")}{check existence of file.R}
\command{file.info("file.R")}{information about file.R}
\command{file.info("file.R")\$size}{specific info (size)}
\command{file.rename("file.R","file2.R")}{rename to file2.R}
\command{file.copy("file.R","file2.R")}{copy}
\command{file.path("file.R")}{relative path to file.R}
\command{file.path("dir1",dir2)}{construct path dir1/dir2}
\command{dir.create("dir1")}{create dir1}
\command{dir.create(file.path("d1","d2",recursive=T))}{}
\command{.}{created nested d1/d2}
%----------------------------------------------------------------------------------------
%	HEADING THREE
%----------------------------------------------------------------------------------------	
\sectiontitle{Sequences}
\command{1:20}{1,2,3,...20}
\command{pi:20}{ 3.14,4.14,5.14,...19.14}
\command{15:1}{15,14,13,...1}
\command{seq(1,10,by=0.5)}{1,1.5,2,...10}
\command{seq(2,10,length=5)}{2,4,6,8,10}
\command{length(x)}{length N of vector x}
\command{1:length(x)}{1,2,3,...N}
\command{seq(along.with=x)}{1,2,3,...N}
\command{seq\_along(x)}{1,2,3,...N}
\command{rep(0,times=5)}{(0 0 0 0 0)}
\command{rep(c(0,1),times=2)}{(0 1 0 1)}
\command{rep(c(0,1),each=2)}{(0 0 1 1)}
\sectiontitle{Vectors: logical and text}
\texttt{
	\textgreater\,  x<-c(1,2,3) \hfill 
	\textgreater\, str<-c("Hello","world") \hfill \\
}
\command{<;>;<=;>=;==;!=}{logical operators}

%----------------------------------------------------------------------------------------
\end{minipage} % End the first column of text
} % End the first division of the page
%----------------------------------------------------------------------------------------
%	SECOND COLUMN SPECIFICATION 
%----------------------------------------------------------------------------------------
\put(105,200){ % Divide the page
\begin{minipage}[t]{85mm} % Create a box to house text
%----------------------------------------------------------------------------------------
%	HEADING FOUR
%----------------------------------------------------------------------------------------
\command{|;\&;!}{OR,AND,NOT}
\command{||;\&\&}{evaluates just the 1st element}
\command{x>1}{FALSE TRUE TRUE}
\command{5>8||6!=8\&\&4>3.9}{TRUE (AND before OR)}	
\command{isTRUE(6>4)}{TRUE}
\command{xor(TRUE,FALSE)}{TRUE (exclusive or)}
\command{which(x>2)}{2 3}
\command{any(x>2)}{TRUE}
\command{all(x>2)}{FALSE}
\command{paste(str,collapse=" ")}{"Hello world"}
\command{str<-c(str,"!")}{"Hello","world","!"}
\command{paste("Hello","!")}{"Hello !"}
\command{paste(x,str,sep="")}{"1Hello" "2world","3!"}
\command{paste(str,1:3,sep="-")}{"Hello-1" "world-2" "!-3"}
%----------------------------------------------------------------------------------------
%	HEADING FIVE
%---------------------------------------------------------------------------------------	
\sectiontitle{Missing valuesss} % Heading five
\texttt{
	\textgreater\, x<-c(1,NA,3) \\
}
\command{x*2}{2 NA 6}
\command{is.na(x)}{FALSE TRUE FALSE}
\command{sum(is.na(x))}{1}
\command{rnorm(n,mean-0,sd=1)}{normal distribution}
\command{sample(x,10,replace=T)}{sample 10 items}
\command{0/0}{NaN (Not a Number)}
\command{Inf-Inf}{NaN (Not a Number)}
\sectiontitle{Subsets}
\texttt{
	\textgreater \, x<-c(7,NA,13) \hfill 
	\textgreater \, y<-c(17,19,23)\hfill\\ 
	\textgreater \, z<-c("foo","bar","qq")
}\\
\command{x[2:3]}{NA 13}
\command{x[is.na(x)]}{NA}
\command{x[!is.na(x)]}{7 13}
\command{x[y>18]}{NA 13}
\command{x[!is.na(x) \& y>18]}{13}
\command{x[c(1,3)]}{7 13}
\command{x[c(-2,-3)]}{7}
\command{x[10]}{NA}
\command{names(y)<-z}{foo|17 bar|19 qq|23}
\command{y["bar"]}{bar|19}
\command{a<-c(foo=17,bar=19,qq=23)}{foo|17 bar|19 qq|23}
\command{identical(a,y)}{TRUE}
\command{M[,11:17]}{subset matrix columns 11:17}
\command{range(y)}{17 23}
\command{unique(c(1,1,2,2,3))}{1 2 3}
\sectiontitle{Matrices}
\texttt{
	\textgreater\,  x<-1:20 \hfill 
	\textgreater\, y<-c("A","B","C","D") \hfill \\
}
\command{dim(x)<-c(4,5)}{create 4x5 matrix}

%----------------------------------------------------------------------------------------
\end{minipage} % End the second column of text
} % End the second division of the page
%----------------------------------------------------------------------------------------
%	THIRD COLUMN SPECIFICATION 
%----------------------------------------------------------------------------------------
\put(200,200){ % Divide the page
\begin{minipage}[t]{85mm} % Create a box to house tex

%----------------------------------------------------------------------------------------
%	IMPORTANT FILES
%----------------------------------------------------------------------------------------
\command{dim(x)}{4 5}
\command{attributes(x)}{\$dim 4 5}
\command{class(x)}{"matrix"}
\command{str(M)}{display internal structure}
\command{matrix(1:20,nrow=4,ncol=5)}{identical to x}
\command{cbind(x,y)}{add column (all characters)}
\command{M<-data.frame(x,y)}{add col (mixed int\&char)}
\command{colnames(M)<-}{name columns}
\command{rownames(M)<-}{name rows}
\command{colMeans(M[1:4])}{column 1-4 means}
\sectiontitle{Functions}
\command{function(arg1,arg2)expr}{basic syntax}
\command{function(x,y)x+y}{sum var x and y}
\command{function(x,y=1)x+y}{default value for y}
\command{args(function)}{show function arguments}
\command{f<-function(fun,x)fun(x)}{pass function as arg}
\command{f(function(x)x+1,6)}{anonymous function}
\command{(...)}{one or more R objects}
\command{function(...)\{paste(">",...,"<")\}}{>...<}
\command{paste(...,sep=" ")}{args after ... must be default}
\command{list1[["arg"]]}{extract named argument}
\command{"\%t\%"<-function(x,y){x*y}}{binary operator}
\command{4 \%t\% 6}{24}
\command{lapply(list,fun)}{applies function to list}
\command{sapply(list,fun)}{returns vector/matrix}
\command{lapply(x,function(arg)expr)}{apply anon fun}
\command{vapply(x,f,numeric(1))}{expect elements num of length 1}
\command{table(x,y)}{counts at each combi}
\command{tapply()}{split data into groups}
\command{tapply(X,INDEX,fun)}{fun(X) for each INDEX }

\sectiontitle{Simulation}
\command{sample(1:6,4,replace=T)}{roll 4 dices}
\command{sample(0:1,10,rep=T,pro=c(0.4,0.6))}{false coin}
\command{rbinom(n,size,prob)}{binomial distribution}
\command{rnorm(n,mean=0,sd=1)}{normal distribution}
\command{rpois(n,lambda)}{discrete poisson}
\command{replicate(n,expr)}{perform expr n times}
\command{hist(x)}{plot histogram}
%----------------------------------------------------------------------------------------
%	FOOTNOTE
%----------------------------------------------------------------------------------------


\linethickness{0.2mm} % Thickness of the footer line
{\color{black}\line(1,0){30}} % Print the line with a custom color

\footnotesize{
Created by Raoul Grouls, 2017\\
verbatim of swirl package
\url{http://swirlstats.com/}
				
Released under the MIT license.}

%----------------------------------------------------------------------------------------

\end{minipage} % End the third column of text
} % End the third division of the page
\end{picture} % End the container for the entire page

%----------------------------------------------------------------------------------------

\end{document}
